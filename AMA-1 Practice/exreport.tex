\PassOptionsToPackage{unicode=true}{hyperref} % options for packages loaded elsewhere
\PassOptionsToPackage{hyphens}{url}
%
\documentclass[]{article}
\usepackage{lmodern}
\usepackage{amssymb,amsmath}
\usepackage{ifxetex,ifluatex}
\usepackage{fixltx2e} % provides \textsubscript
\ifnum 0\ifxetex 1\fi\ifluatex 1\fi=0 % if pdftex
  \usepackage[T1]{fontenc}
  \usepackage[utf8]{inputenc}
  \usepackage{textcomp} % provides euro and other symbols
\else % if luatex or xelatex
  \usepackage{unicode-math}
  \defaultfontfeatures{Ligatures=TeX,Scale=MatchLowercase}
\fi
% use upquote if available, for straight quotes in verbatim environments
\IfFileExists{upquote.sty}{\usepackage{upquote}}{}
% use microtype if available
\IfFileExists{microtype.sty}{%
\usepackage[]{microtype}
\UseMicrotypeSet[protrusion]{basicmath} % disable protrusion for tt fonts
}{}
\IfFileExists{parskip.sty}{%
\usepackage{parskip}
}{% else
\setlength{\parindent}{0pt}
\setlength{\parskip}{6pt plus 2pt minus 1pt}
}
\usepackage{hyperref}
\hypersetup{
            pdftitle={AMA-1 Technical Report: Concentration of protein in blood},
            pdfauthor={Danyu Zhang, Daniel Alonso},
            pdfborder={0 0 0},
            breaklinks=true}
\urlstyle{same}  % don't use monospace font for urls
\usepackage[margin=1in]{geometry}
\usepackage{color}
\usepackage{fancyvrb}
\newcommand{\VerbBar}{|}
\newcommand{\VERB}{\Verb[commandchars=\\\{\}]}
\DefineVerbatimEnvironment{Highlighting}{Verbatim}{commandchars=\\\{\}}
% Add ',fontsize=\small' for more characters per line
\usepackage{framed}
\definecolor{shadecolor}{RGB}{248,248,248}
\newenvironment{Shaded}{\begin{snugshade}}{\end{snugshade}}
\newcommand{\AlertTok}[1]{\textcolor[rgb]{0.94,0.16,0.16}{#1}}
\newcommand{\AnnotationTok}[1]{\textcolor[rgb]{0.56,0.35,0.01}{\textbf{\textit{#1}}}}
\newcommand{\AttributeTok}[1]{\textcolor[rgb]{0.77,0.63,0.00}{#1}}
\newcommand{\BaseNTok}[1]{\textcolor[rgb]{0.00,0.00,0.81}{#1}}
\newcommand{\BuiltInTok}[1]{#1}
\newcommand{\CharTok}[1]{\textcolor[rgb]{0.31,0.60,0.02}{#1}}
\newcommand{\CommentTok}[1]{\textcolor[rgb]{0.56,0.35,0.01}{\textit{#1}}}
\newcommand{\CommentVarTok}[1]{\textcolor[rgb]{0.56,0.35,0.01}{\textbf{\textit{#1}}}}
\newcommand{\ConstantTok}[1]{\textcolor[rgb]{0.00,0.00,0.00}{#1}}
\newcommand{\ControlFlowTok}[1]{\textcolor[rgb]{0.13,0.29,0.53}{\textbf{#1}}}
\newcommand{\DataTypeTok}[1]{\textcolor[rgb]{0.13,0.29,0.53}{#1}}
\newcommand{\DecValTok}[1]{\textcolor[rgb]{0.00,0.00,0.81}{#1}}
\newcommand{\DocumentationTok}[1]{\textcolor[rgb]{0.56,0.35,0.01}{\textbf{\textit{#1}}}}
\newcommand{\ErrorTok}[1]{\textcolor[rgb]{0.64,0.00,0.00}{\textbf{#1}}}
\newcommand{\ExtensionTok}[1]{#1}
\newcommand{\FloatTok}[1]{\textcolor[rgb]{0.00,0.00,0.81}{#1}}
\newcommand{\FunctionTok}[1]{\textcolor[rgb]{0.00,0.00,0.00}{#1}}
\newcommand{\ImportTok}[1]{#1}
\newcommand{\InformationTok}[1]{\textcolor[rgb]{0.56,0.35,0.01}{\textbf{\textit{#1}}}}
\newcommand{\KeywordTok}[1]{\textcolor[rgb]{0.13,0.29,0.53}{\textbf{#1}}}
\newcommand{\NormalTok}[1]{#1}
\newcommand{\OperatorTok}[1]{\textcolor[rgb]{0.81,0.36,0.00}{\textbf{#1}}}
\newcommand{\OtherTok}[1]{\textcolor[rgb]{0.56,0.35,0.01}{#1}}
\newcommand{\PreprocessorTok}[1]{\textcolor[rgb]{0.56,0.35,0.01}{\textit{#1}}}
\newcommand{\RegionMarkerTok}[1]{#1}
\newcommand{\SpecialCharTok}[1]{\textcolor[rgb]{0.00,0.00,0.00}{#1}}
\newcommand{\SpecialStringTok}[1]{\textcolor[rgb]{0.31,0.60,0.02}{#1}}
\newcommand{\StringTok}[1]{\textcolor[rgb]{0.31,0.60,0.02}{#1}}
\newcommand{\VariableTok}[1]{\textcolor[rgb]{0.00,0.00,0.00}{#1}}
\newcommand{\VerbatimStringTok}[1]{\textcolor[rgb]{0.31,0.60,0.02}{#1}}
\newcommand{\WarningTok}[1]{\textcolor[rgb]{0.56,0.35,0.01}{\textbf{\textit{#1}}}}
\usepackage{longtable,booktabs}
% Fix footnotes in tables (requires footnote package)
\IfFileExists{footnote.sty}{\usepackage{footnote}\makesavenoteenv{longtable}}{}
\usepackage{graphicx,grffile}
\makeatletter
\def\maxwidth{\ifdim\Gin@nat@width>\linewidth\linewidth\else\Gin@nat@width\fi}
\def\maxheight{\ifdim\Gin@nat@height>\textheight\textheight\else\Gin@nat@height\fi}
\makeatother
% Scale images if necessary, so that they will not overflow the page
% margins by default, and it is still possible to overwrite the defaults
% using explicit options in \includegraphics[width, height, ...]{}
\setkeys{Gin}{width=\maxwidth,height=\maxheight,keepaspectratio}
\setlength{\emergencystretch}{3em}  % prevent overfull lines
\providecommand{\tightlist}{%
  \setlength{\itemsep}{0pt}\setlength{\parskip}{0pt}}
\setcounter{secnumdepth}{0}
% Redefines (sub)paragraphs to behave more like sections
\ifx\paragraph\undefined\else
\let\oldparagraph\paragraph
\renewcommand{\paragraph}[1]{\oldparagraph{#1}\mbox{}}
\fi
\ifx\subparagraph\undefined\else
\let\oldsubparagraph\subparagraph
\renewcommand{\subparagraph}[1]{\oldsubparagraph{#1}\mbox{}}
\fi

% set default figure placement to htbp
\makeatletter
\def\fps@figure{htbp}
\makeatother


\title{AMA-1 Technical Report: Concentration of protein in blood}
\author{Danyu Zhang, Daniel Alonso}
\date{February 10th, 2021}

\begin{document}
\maketitle

\hypertarget{introduction}{%
\section{Introduction}\label{introduction}}

Our objective is to determine which SNPs are related to concentration.
SNPs are specific sections of genetic code transcribed to string format
(i.e. \emph{AGTGCTGATCGA}).

Here we have a preview of the dataset:

\begin{longtable}[]{@{}lrrrrrrrrr@{}}
\caption{Genetic data}\tabularnewline
\toprule
id & conc & snp1 & snp2 & snp3 & snp4 & snp5 & snp6 & snp7 &
snp8\tabularnewline
\midrule
\endfirsthead
\toprule
id & conc & snp1 & snp2 & snp3 & snp4 & snp5 & snp6 & snp7 &
snp8\tabularnewline
\midrule
\endhead
id1 & 33.49489 & 2 & 3 & 3 & 1 & 2 & 1 & 3 & 3\tabularnewline
id2 & 30.53090 & 1 & 2 & 3 & 2 & 1 & 2 & 2 & 2\tabularnewline
id3 & 31.60567 & 2 & 1 & 2 & 3 & 1 & 3 & 3 & 3\tabularnewline
id4 & 20.90570 & 1 & 1 & 3 & 3 & 3 & 3 & 2 & 2\tabularnewline
id5 & 32.03528 & 2 & 3 & 1 & 1 & 2 & 2 & 2 & 2\tabularnewline
id6 & 29.79114 & 3 & 1 & 2 & 1 & 2 & 2 & 3 & 1\tabularnewline
\bottomrule
\end{longtable}

The dataset contains the following variables:

\begin{itemize}
\tightlist
\item
  \textbf{id}: refers to the ID of the individual who's genetic sequence
  was analyzed
\item
  \textbf{conc}: concentration
\item
  \textbf{snp1} through \textbf{snp8}: specific sections of the genetic
  code analyzed (SNPs)
\end{itemize}

\hypertarget{basic-exploratory-data-anaysis}{%
\section{Basic exploratory data
anaysis}\label{basic-exploratory-data-anaysis}}

We will take a look at the data graphically in a very simple manner:

\hypertarget{distribution-of-concentration}{%
\subsection{\texorpdfstring{Distribution of
\emph{concentration}}{Distribution of concentration}}\label{distribution-of-concentration}}

\includegraphics{./figures/unnamed-chunk-3-1.pdf}

~

We can see that concentration approaches a normal distribution. It would
be reasonable to attempt to fit it to a normal distribution:

~

\includegraphics{./figures/unnamed-chunk-4-1.pdf}

~

And we can see that yes, we can somewhat comfortably say that the
concentration comes from a normal distribution.

\hypertarget{most-dominant-snp-values}{%
\subsection{Most dominant SNP values}\label{most-dominant-snp-values}}

\includegraphics{./figures/unnamed-chunk-5-1.pdf}

~

As for the SNPs, we can see that the most dominant value for the SNPs is
\textbf{2}, while for \emph{snp1} and \emph{snp2} the value \textbf{3}
seems to be slightly more common than the value \textbf{1}.

For the rest of the SNPs, \textbf{2} is the most common value, while
\textbf{1} and \textbf{3} seem quite even.

\newpage

\hypertarget{method}{%
\section{Method}\label{method}}

We are going to use multiple regression using as predictors all other
variables minus \emph{conc} in order to predict our target variable
which is \emph{conc}, we expect to obtain numerical results as our
problem is to predict concentration.

We can see that only the first and the second variables are significant
by checking the p-values, additionaly, the mean squared error is around
26, which is acceptable.

\footnotesize

\begin{verbatim}
#> 
#> Call:
#> lm(formula = conc ~ ., data = trainset)
#> 
#> Residuals:
#>      Min       1Q   Median       3Q      Max 
#> -12.5477  -4.1447   0.6243   4.3540  11.8014 
#> 
#> Coefficients:
#>             Estimate Std. Error t value Pr(>|t|)    
#> (Intercept) 31.88895    6.16232   5.175 2.69e-06 ***
#> snp1         3.22962    0.97608   3.309  0.00158 ** 
#> snp2         1.94927    1.02505   1.902  0.06194 .  
#> snp3        -1.42215    0.97374  -1.461  0.14928    
#> snp4        -1.90791    1.04548  -1.825  0.07291 .  
#> snp5        -1.93077    1.04101  -1.855  0.06847 .  
#> snp6         0.08094    0.93123   0.087  0.93102    
#> snp7        -1.61372    1.10379  -1.462  0.14888    
#> snp8         0.33321    0.95093   0.350  0.72724    
#> ---
#> Signif. codes:  0 '***' 0.001 '**' 0.01 '*' 0.05 '.' 0.1 ' ' 1
#> 
#> Residual standard error: 5.678 on 61 degrees of freedom
#> Multiple R-squared:  0.3416, Adjusted R-squared:  0.2552 
#> F-statistic: 3.955 on 8 and 61 DF,  p-value: 0.0008014
#> [1] 26.80507
\end{verbatim}

\normalsize

\hypertarget{results}{%
\section{Results}\label{results}}

Our most important variables to forecast concentration of protein in
blood of corresponding individual are the genetic profiles ``snp1'' and
``snp2'', and all the other genetic profiles are useless when used to
predict this concentration.

Additionaly, genetic profiles ``snp1'' and ``snp2'' cause positive
effects on the concentration of protein of the individual. Which means
that, increasing genetic profiles ``snp1'' by 1 unit causes an increment
on the concentration by 25 units; and 1 unit of increment on ``snp2''
produces 3.27 of increment on the concentration of protein in the blood.

The model works relatively well as it has mean squared error around 26.
So we can use this model to predict the concentration of protein in the
blood of corresponding individual.

\begin{longtable}[]{@{}lr@{}}
\caption{Variable importance for each SNP}\tabularnewline
\toprule
& Overall\tabularnewline
\midrule
\endfirsthead
\toprule
& Overall\tabularnewline
\midrule
\endhead
snp1 & 3.3087686\tabularnewline
snp2 & 1.9016321\tabularnewline
snp3 & 1.4605048\tabularnewline
snp4 & 1.8249087\tabularnewline
snp5 & 1.8547113\tabularnewline
snp6 & 0.0869160\tabularnewline
snp7 & 1.4619827\tabularnewline
snp8 & 0.3504041\tabularnewline
\bottomrule
\end{longtable}

\hypertarget{bibliography}{%
\section{Bibliography}\label{bibliography}}

Data obtained from
\href{http://portal.uc3m.es/portal/page/portal/dpto_estadistica/personal/Stefano_Cabras}{Stefano
Cabras}

\hypertarget{appendix-code}{%
\section{Appendix: Code}\label{appendix-code}}

Here goes the \emph{literate programming} part.

\footnotesize

\begin{Shaded}
\begin{Highlighting}[]
\CommentTok{# importing libraries}
\KeywordTok{library}\NormalTok{(dplyr)}
\KeywordTok{library}\NormalTok{(MLmetrics)}
\KeywordTok{library}\NormalTok{(caret)}
\KeywordTok{library}\NormalTok{(MASS)}
\KeywordTok{library}\NormalTok{(stringr)}
\KeywordTok{library}\NormalTok{(ggplot2)}
\KeywordTok{library}\NormalTok{(fitdistrplus)}

\CommentTok{# importing data}
\NormalTok{gen <-}\StringTok{ }\KeywordTok{read.csv}\NormalTok{(}\StringTok{"./data/gendata.csv"}\NormalTok{, }\DataTypeTok{header=}\OtherTok{TRUE}\NormalTok{, }\DataTypeTok{sep=}\StringTok{";"}\NormalTok{)}
\NormalTok{knitr}\OperatorTok{::}\KeywordTok{kable}\NormalTok{(}
    \KeywordTok{head}\NormalTok{(gen),}
    \DataTypeTok{booktabs=}\OtherTok{TRUE}\NormalTok{,}
    \DataTypeTok{caption=}\StringTok{"Genetic data"}\NormalTok{,}
\NormalTok{)}
\NormalTok{gen <-}\StringTok{ }\NormalTok{gen[}\DecValTok{2}\OperatorTok{:}\KeywordTok{length}\NormalTok{(gen)]}
\NormalTok{cols <-}\StringTok{ }\KeywordTok{length}\NormalTok{(}\KeywordTok{names}\NormalTok{(gen))}
\NormalTok{gen <-}\StringTok{ }\KeywordTok{sapply}\NormalTok{(gen, as.numeric)}

\CommentTok{# concentration histogram}
\KeywordTok{par}\NormalTok{(}\DataTypeTok{mfrow=}\KeywordTok{c}\NormalTok{(}\DecValTok{1}\NormalTok{,}\DecValTok{1}\NormalTok{))}
\KeywordTok{hist}\NormalTok{(gen[,}\DecValTok{1}\NormalTok{], }\DataTypeTok{main=}\StringTok{"concentration"}\NormalTok{)}

\CommentTok{# SNP plots}
\KeywordTok{par}\NormalTok{(}\DataTypeTok{mfrow=}\KeywordTok{c}\NormalTok{(}\DecValTok{3}\NormalTok{,}\DecValTok{3}\NormalTok{))}
\ControlFlowTok{for}\NormalTok{ (i }\ControlFlowTok{in} \DecValTok{2}\OperatorTok{:}\NormalTok{cols) \{}
    \KeywordTok{barplot}\NormalTok{(}\KeywordTok{table}\NormalTok{(gen[,i]), }\DataTypeTok{main=}\KeywordTok{str_interp}\NormalTok{(}\StringTok{"snp$\{i-1\}"}\NormalTok{))}
\NormalTok{\}}

\CommentTok{# converting gen to dataframe}
\NormalTok{gen <-}\StringTok{ }\KeywordTok{as.data.frame}\NormalTok{(gen)}

\CommentTok{# Splitting dataset into train and test}
\NormalTok{n=}\KeywordTok{nrow}\NormalTok{(gen)}
\KeywordTok{set.seed}\NormalTok{(}\DecValTok{7}\NormalTok{)}
\NormalTok{trainset=(}\DecValTok{1}\OperatorTok{:}\NormalTok{n)}\OperatorTok\KeywordTok{sample}\NormalTok{(n,}\KeywordTok{floor}\NormalTok{(n}\OperatorTok{*}\FloatTok{0.7}\NormalTok{)) }
\NormalTok{testset=}\OperatorTok{!}\NormalTok{trainset }
\NormalTok{trainset=gen[trainset,]}
\NormalTok{testset=gen[testset,]}

\CommentTok{# Modelling}
\NormalTok{model1 <-}\StringTok{ }\KeywordTok{lm}\NormalTok{(conc }\OperatorTok{~}\StringTok{ }\NormalTok{., }\DataTypeTok{data =}\NormalTok{ trainset)}
\KeywordTok{summary}\NormalTok{(model1)}

\CommentTok{# Prediction and Metrics}
\NormalTok{pred1 =}\StringTok{ }\KeywordTok{predict}\NormalTok{(model1, testset)}
\KeywordTok{MSE}\NormalTok{(pred1, testset}\OperatorTok{$}\NormalTok{conc)}

\CommentTok{# variable importance plot}
\NormalTok{impo <-}\StringTok{ }\KeywordTok{varImp}\NormalTok{(model1, }\DataTypeTok{scale=}\OtherTok{FALSE}\NormalTok{)}
\NormalTok{knitr}\OperatorTok{::}\KeywordTok{kable}\NormalTok{(}
\NormalTok{    impo,}
    \DataTypeTok{booktabs=}\OtherTok{TRUE}\NormalTok{,}
    \DataTypeTok{caption=}\StringTok{"Variable importance for each SNP"}\NormalTok{,}
\NormalTok{)}
\end{Highlighting}
\end{Shaded}

\end{document}
